\chapter{Conclusion and Future Work}
The thesis presents a practical approach to real-time pose estimation by combining IMU data with visual information. Using an Extended Kalman Filter as a central component for sensor fusion reflects a creative blend of traditional estimation methods with contemporary sensor data to address the pressing need for reliable localization in environments where GNSS is not available.

The algorithm's capacity to correct its path and align with verified trajectories is a testament to its potential. It indicates a system that's designed to be aware and adaptive, qualities that are essential in dynamic environments where conditions change rapidly.

Emphasizing segment-based mapping and localization demonstrates an appreciation for the finer details of environmental interaction, which is often missed by more general point-based systems. This detail-oriented approach could lead to a deeper understanding of various terrains and settings, a valuable trait for precision-critical applications.

The methods and tools used here suggest a system that can be scaled and tailored to different needs. The versatility of this system opens up possibilities ranging from navigating indoor spaces to maneuvering through cluttered outdoor environments, and beyond.

This work signals the broader impact that thoughtfully designed autonomous systems might have. The strategies and techniques developed could potentially be adapted for diverse uses, including autonomous vehicles and sophisticated navigational aids.

Looking ahead, further refining the sensor calibration process, improving data fusion methods, and considering the inclusion of additional types of sensors are practical steps that could be taken. These efforts, alongside making the system more computationally efficient, will be critical for preparing the system for real-world application where dependability is crucial. Additionally, tapping into the capabilities of deep learning for more effective feature extraction and adding layers of semantic understanding might offer new ways to interpret and interact with different environments.

For real-world applications, the system's robustness and adaptability should be evaluated through comprehensive testing across a range of scenarios. These tests will provide the solid grounding needed for future exploration and potential deployment.

Despite the challenges, such as sensor noise and the limited computing resources, the resilience of the approach is noteworthy. The ability to adapt and maintain a level of stability in the face of such obstacles highlights the practical value of the research.

In sum, this thesis makes a constructive contribution to the domain of GNSS localization and autonomous navigation. The smart combination of IMU data with visual cues, supported by machine learning insights, marks a step towards creating autonomous systems that can operate without GPS. The groundwork laid here is ripe for future advancements that could significantly enhance UAV navigation and even influence the wider field of autonomous system design. The methodologies, while there's room for improvement, set the stage for ongoing exploration in this dynamic area of study.